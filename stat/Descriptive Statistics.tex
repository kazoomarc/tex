\documentclass[12pt,a4paper]{article}

\usepackage{graphicx} % Required for inserting images
\usepackage{setspace} % Required for paragraph setting
\usepackage{amsmath}
\title{DESCRIPTIVE STATISTICS}
\author{Joel Mwala}
\begin{document}
%
\maketitle

\section{Introduction}
\setlength{\parskip}{0.5em} % Set the desired paragraph spacing
\setstretch{1.5} % Set the desired line height between paragraphs
def 1: \textbf{Statistics} is the science of theory and methods of collection, organisation, analysis and interpretation of data sets so as to dertermine their essential characteristics. \\ - There are established theories and approaches under each of the stages of data collection, organisation, analysis, interpretation and dissermination. These theories form a field of study called \underline{statistics}. \\ - For instance underlying the stage of analysis of data is the vast mathematical apparatus of abstract concepts, theories, formulae, algorithm and so on that constitute the statistical tools that are employed to study a data set. \\ - The tools allow an analyst to to build a framework for good decision making.
\newline
def 2: \textbf{Statistics} are mathematical quantities or expressions that are applied on data in order to elicit essential characteristics of the data at hand e.g the statistic for obtaining central value of the sample on the mean given by $\bar{x} = \frac{1}{n}\sum_{i = 1}^{n} x_i$ where $i = 1,2,3,4, \dots,19$ are sample parts. \

def 3: \textbf{Statistics} may also imply the value calculated in def 2.
The third definition is ussually less found or used in formal context.

%end of section
\section{Types of statistics}
\textbf{1 . Descriptive statistics} Includes any treatment of data designed to summarize their essential features.

- Interest is in arranging data in readable form eg using tables, charts and graphs or compute percentage, average, rate of change, variance, etc.

- The aim of such task is to understand the data at hand and not beyond

\textbf{2. Inferential or inductive statistics} is concerned with making estimates, predictions or forecasts and generalization.

- It is  a process of infering something about the whole from an examination of only a part.

- This process is carried out through sampling ie. using representative subgroup of items and conclude about the entire group/ population

- Because conclusions are made by using information from a sample of items not every item, there is some level of error that will most assuredly taint the whole conclusions about the whole items.

- Hence a measure of uncertainity of the surface made most accompanying in generalization.

%end of section
\section{Scales of Measurement}

\subsection{Norminal Scale}
\subsection{Ordinal Scale}
\subsection{Interval Scale}
\subsection{Ratio Scale}

%end of section
\section{Measures of Position}

\subsection{Percentiles}
\begin{equation}
    P_{th} = \frac{(n+1)p}{100}
\end{equation}
\subsection{Quartiles}


%end of section
\section{Measures of Central Tendancy}

\subsection{Median}
\subsection{Mode}
\subsection{Mean}

\begin{equation}
    \bar{x} = \frac{\sum_{i = 1}^{n} x_i}{n}= \frac{x_1 + x_2 + x_3 + \ldots + x_n}{n}
\end{equation}

\begin{equation}
    \mu = \frac{\sum_{i = 1}^{N} x_i}{N}= \frac{x_1 + x_2 + x_3 + \ldots + x_N}{N}
\end{equation}


%end of section
\section{Measures of Variability / Dispersion}

\subsection{Range}
\subsection{Inter-Quartile Range (IQR)}
\subsection{Variance}
\subsection{Relative Variance and Coefiecient of Variation}
\subsection{Recursion formulae for sample mean and variance}


%end of section
\section{Measures of Symmetry}
\subsection{Skewness}
\subsection{Kurtosis}


\end{document}