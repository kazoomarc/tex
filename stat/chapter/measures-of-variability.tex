\chapter{Measures of Variability / Dispersion}%!chapter
The variance, range and interquartile range are some of the measurement of dispersion of a dataset.

They quantify amount of spread of data from the central value.
\section{Range}
NOT AVAILABLE
\section{Inter-Quartile Range (IQR)}
\textbf{Definition:} An interquartile range (IRQ) is the difference between the first and third quartiles.

The interquartile range is a measure of spread of the data.

IQR tells you the spread of the middle half of your distribution.

IQR contains the {difference between} second and third quartiles  or the middle half that data

IQR can be used to find outliers in data.

\section{Outlier analysis}

Outlier observations are the ones that do not confirm to distribution structure of the rest of the data.

The outliers can be identified by the following formulae:

\begin{equation}
    < Q_1 - 1.5\text{IQR or} > Q_3 + 1.5\text{IQR}
\end{equation}
\section{Variance}
Sample variance of $X$ is
\begin{equation}
    S^2 = \frac{\sum_{i = 1}^{n}(x_i - \bar{x})^2}{n - 1} = \frac{\sum_{i = 1}^{n} x_i^2 - \frac{(\sum_{i = 1}^{n}x_i)^2}{n}}{n-1}
\end{equation}

It is average squared deviations of observations from the mean.

The square root of variance is called standard deviation, its symbol is $\delta$ for population data and $s$ for sample data

\begin{equation}
    S_x = \sqrt[2]{\frac{\sum_{i = 1}^{n} x_i^2 - \frac{(\sum_{i = 1}^{n}x_i)^2}{n}}{n-1}}
\end{equation}
\begin{equation}
    \delta = \sqrt[2]{\delta^2} = \sqrt[2]{\frac{\sum_{i = 1}^{N} x_i^2 - \frac{(\sum_{i = 1}^{N}x_i)^2}{N}}{N}}
\end{equation}

The standard deviation measures standard distance of each data point from their mean.
\section{Relative Variance and Coefiecient of Variation}

Relative variance is the ratio of variance of the data to the square of the mean of the data set. ie,
\begin{center}
    for sample mean
\end{center}
\begin{equation}
    V_{\bar{x}}^2 = \frac{S_x^2}{(\bar{X}^2)}
\end{equation}
\begin{center}
    OR
\end{center}
\begin{center}
    For population
\end{center}
\begin{equation}
    V_{\mu}^2 = \frac{\delta_x^2}{(\mu_x^2)}
\end{equation}

It is mostly used to compare variation across data sets, as $V_{\mu}^2$ or $V_{\bar{x}}^2$ has no units.

The square root of relative variance is called the coeficient of variation(Cv) or "relative error" of mean $\bar{x}$ or $\mu$

It is also dimensionless quantity and ussually given in percentages.

\begin{center}
    for sample mean
\end{center}
\begin{equation}
    CV_{\bar{x}} = \frac{S_x}{\bar{X}}
\end{equation}
\begin{center}
    OR
\end{center}
\begin{center}
    For population
\end{center}
\begin{equation}
    CV_{\mu} = \frac{\delta_x}{\mu_x}
\end{equation}


Eg: Given the sample data 1,2,3,4,5. Compute mean, variance, standard deviation, relative variance and coeficient of variation(CV)for the data.

Solution:

\begin{enumerate}
    \item[(a)]

        \begin{align*}
            \bar{x} & = \frac{\sum_{i = 1}^{n} x_i}{n}               \\
                    & n = 5, x_1 = 1 x_2 = 2 x_3 = 3 x_4 = 4 x_5 = 5 \\
            \bar{x} & = \frac{1 + 2 + 3 + 4 + 5}{5}                  \\
            \bar{x} & = \frac{15}{5}                                 \\
            \bar{x} & = 3                                            \\
        \end{align*}
        Interpretation: The data values have higher weight around the value of $3$ and they are concentrated around the value of $3$
    \item[(b)]
        \begin{align*}
            S_x^2 & = \frac{\sum_{i = 1}^{n}x_i^2 - \frac{(\sum_{i = 1}^{n}x_i)^2}{n}}{n-1}         \\
            S_x^2 & = \frac{(1^2 + 2^2 + 3^2 + 4^2 + 5^2) - \frac{(1 + 2 + 3 + 4 + 5)^2}{5}}{5 - 1} \\
            S_x^2 & = \frac{55 - 45}{4}                                                             \\
            S_x^2 & = 2.5                                                                           \\
        \end{align*}

    \item[(c)] \begin{align*}
            S_x & = \sqrt[2]{S_x^2} \\
            S_x & = \sqrt[2]{2.5}   \\
            S_x & = 1.581           \\
        \end{align*}


        Interpretation: The result means each value in the data set is on average at a distance of $1.581$ away from mean.
    \item[(d)]
        \begin{align*}
            V_x^2 & = \frac{S_x^2}{(\bar{x})^2} \\
            V_x^2 & = \frac{2.5}{3^2}           \\
            V_x^2 & = \frac{2.5}{9}             \\
            V_x^2 & = 0.278                     \\
        \end{align*}
    \item[(e)]
        \begin{align*}
            C_v & = \sqrt[2]{V_x^2}                     \\
            C_v & = \sqrt[2]{\frac{S_x^2}{(\bar{x})^2}} \\
            C_v & = \frac{S_x}{\bar{x}}                 \\
            C_v & = \frac{1.581}{3}                     \\
            C_v & = 0.527                               \\
            C_v & = 52.7\%                              \\
        \end{align*}
        The coeficient of variation result shows that that the ratio of standard deviation to the sample mean is $52.7\%$
\end{enumerate}
\section{Recursion formulae for sample mean and variance}
- We can compute mean of $(n + 1)$ data poinsts effectively from knowing the mean ($\bar{x}$) of the first $n$ of them through a recursion formula for mean.
- Suppose we know $\bar{x}_{n}$ (ie average of $x_1, x_2, x_3, \dots, x_n$) and another observation $x_{n+1}$ becomes available then;

\begin{equation}
    \bar{x}_{n + 1} = \frac{x_{n+1} + n\bar{x}_n}{n + 1}
\end{equation}

ie, the result for $(n + 1)^{st}$ term is obtained from the value of the $n^{th}$ terms.
eg. suppose the mean of a set of $n = 50$ values of some variable $x$ in $\bar{x}_n = \bar{x}_{50} = 63.75$. Let us now assume that a fifty-first observation $x_{n + 1} = x_{61} = 22$ becomes available. What is the average of all data points?

solution



\begin{align*}
    \bar{x}_{n + 1} & = \frac{x_{n+1} + n\bar{x}_n}{n + 1} \\
                    & = \frac{22 + 50(63.75)}{50 + 1}      \\
                    & = \frac{3209.5}{51}                  \\
                    & = 62.93                              \\
\end{align*}

Now for variance, suppose we know $S_n^2$ (ie. the variance of the values $x_1, x_2, \ldots, x_n$) and another observation $x_{n + 1}$ becomes available.

Given
\begin{center}
    $S_n^2 = \frac{\sum_{i = 1}^{n}(x_i - \bar{x}_n)^2}{n - 1}$ then ; \\

    $S_{n + 1}^2 = \frac{(x_{n+1} - \bar{x}_n)^2}{n} + \frac{(n - 1)S_n^2}{n}$
\end{center}

Eg. Suppose mean and variance of a set of 25 values of some variable $x$ are $\bar{x}_n = \bar{x}_{25} = 10$ and $S_n^2 = S_{25}^2 = 8.72$ respectively. Assume a twenty-sixth observation $x_{n+1} = x_26 = 14$ is obtained. what is the variance of all $26$ data points?

Solution:
\begin{align*}
    S_{n + 1}^2 & = \frac{(x_{n+1} - \bar{x}_n)^2}{n} + \frac{(n - 1)S_n^2}{n} \\
    S_{26}^2    & = \frac{(14 - 10)^2}{25} + \frac{(25 - 1)8.72}{25}           \\
    S_{26}^2    & = 9.0112                                                     \\
\end{align*}



\newpage
%end of chapter
