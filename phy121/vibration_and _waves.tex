\documentclass{article}

\usepackage{graphicx} % Required for inserting images
\usepackage{hyperref} % Required for the links to work%
\usepackage{setspace}
\usepackage{enumerate} %Required for the lists
\usepackage[left = 0.8in, right = 0.8in, top = 1in, bottom = 1in]{geometry} % Required for setting page sizes

\title{\textbf{PHY 121 \\[1in] VIBRATIONS, WAVES and OPTICS} }
\date{\today}
\author{Dr J.S.P MLATHO}


\begin{document}
\maketitle
\vspace{4in}
\begin{center}
    compiled into \LaTeX by \href{https://www.joelmwala.tech}{  {JOEL MWALA}}
\end{center}
\newpage

\doublespacing
\tableofcontents
\singlespacing

\newpage



\section{VIBRATIONS}
\subsection{Vibrations}
A vibration is a to and fro movement of an object over the same path. Examples where such motions can be observed are

\begin{enumerate}[(A)]
    \item Pendulum
    \item Mass suspended at the end of the stpring
\end{enumerate}


\newpage






\section{Stroboscope}

\begin{itemize}
    \item \textbf{\texttt{Definition}}: A stroboscope is an instrument used for studying periodic
          motion or determining speeds of rotation by shining a bright light at
          intervals so that a moving or rotating object appears stationary
    \item Normally, it consists of a lamp which produces brief repetitive flashes of light at a particular frequency
    \item Normally, a stroboscope consists of a lamp which produces brief
          repetitive flashes of light at a frequency
    \item The rate of the stroboscope is adjustable to different frequencies
          Thus, when a vibrating object is observed with the stroboscope at a
          frequency of its vibration (or multiple of it) it appears stationery
    \item Therefore, stroboscope is also used to measure frequency of
          vibrating or oscillating objects
\end{itemize}

\newpage






\section{SUPERPOSITION OF WAVES}
\subsection{Superposition of waves}
\subsection{properties of wave motion}

\newpage


\section{Stationary waves}
\newpage


\section{Properties of wave motions}
\subsection{Reflection of water waves}
\subsection{Diffraction of water waves}
\subsection{Interference of water waves}

\newpage


\section{Interference of waves}
\newpage


\section{Optics}
\subsection{Reflection of light – laws of reflection}
\subsection{Formation of Image due to reflection}
\subsection{Plane mirrors}
\subsection{Curved mirrors – mirror formula}
\subsection{Concave mirrors}
\subsection{Convex mirrors}
\subsection{Refraction of light}
\subsection{Laws of refraction}
\subsection{Formulation images due to refraction lens –lens formula}
\subsection{Convex lenses}
\subsection{Concave lenses}
\subsection{ Dispersion of light – prisms}

\newpage


\section{Images in spherical mirrors}
\newpage


\section{Refraction air plane surface}
\newpage



























\section{Recommended Reading}

\begin{enumerate}
    \item Fundamentals of Physics 7th Edition Halliday Resnick Walker -Chapter 16, 17, 18: page 495 \\
    \item College Physics Serway 7th Edition - Chapter 13, 14
\end{enumerate}
\newpage


\end{document}