\documentclass{article}
\usepackage{graphicx} % Required for inserting images
\usepackage{amsmath}
\usepackage{amssymb}

\usepackage{amsthm}
\newtheorem{question}{}
\usepackage{enumitem}
\usepackage{tasks}

\usepackage[a4paper]{geometry}
% Set the page margins
\geometry{
    left=3cm,
    right=3cm,
    top=3cm,
    bottom=3cm,
}

\usepackage{tikz}%border around the page
\usepackage{eso-pic}%put every tikz border around every page
\usetikzlibrary{calc}
\AddToShipoutPictureBG{%
    \begin{tikzpicture}[overlay,remember picture]
        \draw[line width=1.5pt]
        ($ (current page.north west) + (1.5cm,-1.5cm) $)
        rectangle
        ($ (current page.south east) + (-1.5cm,1.5cm) $);
        \node at ($ (current page.north west) + (1.7cm, -2.5cm) $) [anchor=west] {Student Name: \makebox[20 em]{\dotfill}};
        \node at ($ (current page.north east) + (-1.7cm,-2.5cm) $) [anchor=east] {Registration No.: \makebox[14 em]{\dotfill}};
        \node at ($ (current page.south west) + (1.7cm,2cm) $) [anchor=west] {MAT123};
        \node at ($ (current page.south) + (0cm, 2cm) $) [anchor=center] {The only way to do maths is to do maths};
        \node at ($ (current page.south east) + (-1.7cm,2cm) $) [anchor=east] {\thepage};
    \end{tikzpicture}%
}

% Define the maximum number of dots for the registration number
%\newcommand{\maxdots}{25}



\usepackage{fancyhdr}
\pagestyle{empty}
\fancyhf{} % Clear default header/footer
%\lhead{Student Name:\makebox[20 em]{\dotfill}}
%\rhead{Registration No.: \makebox[14 em]{\dotfill}}
%\rfoot{\thepage}
%\cfoot{\begin{tabular}{c}The only way to do maths is to do maths\end{tabular}}
%\lfoot{MAT123}

\usepackage{lipsum} % For placeholder text


\title{Title}
\author{First LastName}

\begin{document}

%question 1
\begin{question}
    Convert the following from degree to radian measure:
    \begin{tasks}
        \task task1
        \vspace{7 cm}
        \task task1
        \vspace{7 cm}
    \end{tasks}
\end{question}

%question 2
\begin{question}
    Convert the following from degree to radian measure:
    \begin{tasks}
        \task task1
        \vspace{7 cm}
        \task task1
        \vspace{7 cm}
    \end{tasks}
\end{question}

%question 3
\begin{question}
    Convert the following from degree to radian measure:
    \begin{tasks}
        \task task1
        \vspace{7 cm}
        \task task1
        \vspace{7 cm}
    \end{tasks}
\end{question}

%question 4
\begin{question}
    Convert the following from degree to radian measure:
    \begin{tasks}
        \task task1
        \vspace{7 cm}
        \task task1
        \vspace{7 cm}
    \end{tasks}
\end{question}

%question 5
\begin{question}
    Convert the following from degree to radian measure:
    \begin{tasks}
        \task task1
        \vspace{7 cm}
        \task task1
        \vspace{7 cm}
    \end{tasks}
\end{question}

%question 6
\begin{question}
    Convert the following from degree to radian measure:
    \begin{tasks}
        \task task1
        \vspace{7 cm}
        \task task1
        \vspace{7 cm}
    \end{tasks}
\end{question}

%question 7
\begin{question}
    Convert the following from degree to radian measure:
    \begin{tasks}
        \task task1
        \vspace{7 cm}
        \task task1
        \vspace{7 cm}
    \end{tasks}
\end{question}

%question 8
\begin{question}
    Convert the following from degree to radian measure:
    \begin{tasks}
        \task task1
        \vspace{7 cm}
        \task task1
        \vspace{7 cm}
        \task task1
        \vspace{7 cm}
        \task task1
        \vspace{7 cm}
    \end{tasks}
\end{question}

%question 9
\begin{question}
    Convert the following from degree to radian measure:
    \begin{tasks}
        \task task1
        \vspace{7 cm}
        \task task1
        \vspace{7 cm}
    \end{tasks}
\end{question}

\end{document}