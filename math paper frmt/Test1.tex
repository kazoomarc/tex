\documentclass{article}
\usepackage{graphicx} % Required for inserting images
\usepackage{amsmath}
\usepackage{amssymb}

\usepackage{amsthm}


\newtheoremstyle{custom}
    {\topsep}   % Space above
    {\topsep}   % Space below
    {}          % Body font (empty = "normal" font)
    {}          % Indent amount (empty = no indent, \parindent = para indent)
    {\bfseries} % Theorem head font
    {.}         % Punctuation after theorem head
    { }         % Space after theorem head
    {}          % Theorem head spec (can be left empty, meaning `normal')

\theoremstyle{custom}
\newtheorem{question}{}
\usepackage{enumitem}
\usepackage{tasks}

\usepackage[a4paper]{geometry}
% Set the page margins
\geometry{
    left=3cm,
    right=3cm,
    top=3cm,
    bottom=3cm,
}

\usepackage{tikz}%border around the page
\usepackage{eso-pic}%put every tikz border around every page
\usetikzlibrary{calc}
\AddToShipoutPictureBG{%
    \begin{tikzpicture}[overlay,remember picture]
        \draw[line width=1.5pt]
        ($ (current page.north west) + (1.5cm,-1.5cm) $)
        rectangle
        ($ (current page.south east) + (-1.5cm,1.5cm) $);
        \node at ($ (current page.north west) + (1.7cm, -2.5cm) $) [anchor=west] {Student Name: \makebox[20 em]{\dotfill}};
        \node at ($ (current page.north east) + (-1.7cm,-2.5cm) $) [anchor=east] {Registration No.: \makebox[14 em]{\dotfill}};
        \node at ($ (current page.south west) + (1.7cm,2cm) $) [anchor=west] {MAT123};
        \node at ($ (current page.south) + (0cm, 2cm) $) [anchor=center] {"The only way to learn mathematics is to do maths"};
        \node at ($ (current page.south east) + (-1.7cm,2cm) $) [anchor=east] {\thepage};
    \end{tikzpicture}%
}

% Define the maximum number of dots for the registration number
%\newcommand{\maxdots}{25}



\usepackage{fancyhdr}
\pagestyle{empty}
\fancyhf{} % Clear default header/footer
%\lhead{Student Name:\makebox[20 em]{\dotfill}}
%\rhead{Registration No.: \makebox[14 em]{\dotfill}}
%\rfoot{\thepage}
%\cfoot{\begin{tabular}{c}The only way to do maths is to do maths\end{tabular}}
%\lfoot{MAT123}

\usepackage{lipsum} % For placeholder text


\title{Title}
\author{First LastName}

\begin{document}

%question 1
\begin{question}
    Convert the following from degree to radian measure:
    \begin{tasks}
        \task $-270^{\circ}$
        \vspace{7 cm}
        \task $135^{\circ}$
        \vspace{7 cm}
    \end{tasks}
\end{question}

%question 2
\begin{question}
    Convert the following from radian to degree measure:
    \begin{tasks}
        \task $\frac{9\pi}{4}$
        \vspace{7 cm}
        \task $-\frac{5\pi}{6}$
        \vspace{7 cm}
    \end{tasks}
\end{question}

%question 3
\begin{question}
    \begin{tasks}
        \task Find the value of an acute angle $x$, if $\sin{x} = \cos{20^{\circ}}$.
        \vspace{7 cm}
        \task Evalueate the value of $x$, if $\sec(5x) = \csc{x + 18^{\circ}}$, where $5x$ is an acute angle.
        \vspace{7 cm}
    \end{tasks}
\end{question}

%question 4
\begin{question}
    Evaluate the following:
    \begin{tasks}
        \task $\tan{(-\frac{\pi}{4})}$
        \vspace{7 cm}
        \task $\sin{(-\frac{\pi}{6})}$
        \vspace{7 cm}
    \end{tasks}
\end{question}

%question 5
\begin{question}
    Without using a calculator, find the exact values of the following:
    \begin{tasks}
        \task $\tan{(\frac{4\pi}{3})}$
        \vspace{7 cm}
        \task $\cos{-240^{\circ}}$
        \vspace{7 cm}
    \end{tasks}
\end{question}

%question 6
\begin{question}
    \begin{tasks}
        \task Use identities to find $\cos{\theta}$, given that $\sin{\theta} = \frac{3}{5}$ and $\theta$ is in a quadrant II.
        \vspace{7 cm}
        \task proove the following trignometric identity
        \begin{equation}
            \frac{1}{1 - \sin{\theta}} + \frac{1}{1 + \sin{\theta}} = 2 \sec^2 \theta
        \end{equation}
        \vspace{7 cm}
    \end{tasks}
\end{question}

%question 7
\begin{question}
    Graph the following trigonometrical functions:
    \begin{tasks}
        \task $y = 2 \sin{(3\theta)}$
        \vspace{7 cm}
        \task $y = -3\cos{\theta} + 2$
        \vspace{7 cm}
    \end{tasks}
\end{question}

%question 8
\begin{question}
    use any appropriate trignometric identities to find the exact values of the following
    \begin{tasks}
        \task $\sin{(\frac{4\pi}{3})}$
        \vspace{7 cm}
        \task $\cos{\frac{\theta}{2}}$ given that $\cos\theta = -\frac{1}{3}$
        \vspace{7 cm}
        \task $\sin 54.5^{\circ} \cos 7.5^{\circ}$
        \vspace{7 cm}
        \task $\sin 75^{\circ} -  \sin 15^{\circ}$
        \vspace{7 cm}
    \end{tasks}
\end{question}

%question 9
\begin{question}
    Prove the following trignometric identities
    \begin{tasks}
        \task $2 \cos^{2} x = 1 + \cos {2x}$
        \vspace{7 cm}
        \task $\cos{3\theta} = 4 \cos^3 \theta - 3 \cos\theta$ \nobreakspace{2} [Hint: $\cos{3\theta} = \cos{(2\theta + \theta)}$]
        \vspace{7 cm}
    \end{tasks}
\end{question}

\end{document}