\chapter{Types of statistics} %!chapter
\section{Descriptive statistics}
Includes any treatment of data designed to summarize their essential features.

Interest is in arranging data in readable form eg using tables, charts and graphs or compute percentage, average, rate of change, variance, etc.

The aim of such task is to understand the data at hand and not beyond.

\section{Inferential or inductive statistics}
is concerned with making estimates, predictions or forecasts and generalization.
It is  a process of infering something about the whole from an examination of only a part.

This process is carried out through sampling ie. using representative subgroup of items and conclude about the entire group/ population

Because conclusions are made by using information from a sample of items not every item, there is some level of error that will most assuredly taint the whole conclusions about the whole items.

Hence a measure of uncertainity of the surface made most accompanying in generalization.
\newpage
%end of chapter
