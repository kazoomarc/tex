\chapter{Scales of Measurement}%!chapter

\section{Norminal Scale}
It is associated with the word "name" and it identifies categories of data.

observations on the norminal scale posses neither numerical values nor order.
During data ananlysis numerical codes eg $0, 1, 2, 3$ can be assigned to outcomes of the measurement just for identification of category.

Example: sex variable whose outcomes are male or female is a variable on a norminal scale.
When numbers are used to classfy categories or levels of a variable that is on norminal scale, the magnitude of the differencies between those numerical values is meaningless.

This scale of measurement apply to qaulitative type of data or categorical data.

The only valid operations for variables represented by norinal scale are "$=$" or "$\neq$".
\section{Ordinal Scale}
It includes all the properties of norminal scale with additinal that the observations can be ranked from the smallest to the highest from the least important to the most important.

Example, type of car (Salon, Lorry, Bus).
Although the level of car is better than the other the ranking cannot indicate how much better, the numerical values assigned to data points just indicate type/ level of a car inorder of engine performance and the difference between the numerical values are meanings Eg. Type of membership to club:
\begin{enumerate}
    \item Ordinary member
    \item regular member
    \item Executive member
\end{enumerate}
Both norminal and ordinal scales are termed onmetric scales because the differences among their values are of no consequeces.
\section{Interval Scale}
It includes all properties of ordinal scale with additional property that the distance between the observations is meaningful.

The numbers assigned to the observations indicates order and posses the property that the difference between any two consecutive values is the same as the difference between any two other values (ie the difference $10 - 9 = 1$ has the same meaning as $3 - 2 = 1$)

While this scale has a zero point, its location is arbitrary.

The ratios of interval scale have no meaning.
Eg: Temperature is measured on an interval scale. $0^{\circ}C$ does not imply the absence of heat. Also $40^{\circ}C$ is not twice as cold as $20^{\circ}C$

Grades of students in Mathematics is also an example of interval scale variable. A score of $0\%$ does not imply lack of knowledge.

The mathematical operations for handling variables measured measured on an interval scale are "$=$", "$\neq$","$>$","$<$","$+$","$-$".
\section{Ratio Scale}
Includes all the properties of interval scale with an added property that ratios of observations are meaningful. This is the case because absolute zero is uniquely defined.

Eg height of a person $0cm$ means the absence of any height and height of $8cm$ is twice as tall as height of $4cm$ (the ratio $\frac{8}{4} = 2$)

mathematical operations permitted under this scale are :

"$=$", "$\neq$","$>$","$<$","$+$","$-$", "$\times$", "$\div$"

Both Interval and ratio scales are called metric scales (since the difference between values are meaningful)

Variables of these scales are quantitative.
\newpage
%end of chapter
