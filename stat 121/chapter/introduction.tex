\chapter{Introduction} %!chapter
\setlength{\parskip}{0.5em} % Set the desired paragraph spacing
\setstretch{1.5} % Set the desired line height between paragraphs
def 1: \textbf{Statistics} is the science of theory and methods of collection, organisation, analysis and interpretation of data sets so as to dertermine their essential characteristics.

There are established theories and approaches under each of the stages of data collection, organisation, analysis, interpretation and dissermination. These theories form a field of study called \underline{statistics}.

For instance underlying the stage of analysis of data is the vast mathematical apparatus of abstract concepts, theories, formulae, algorithm and so on that constitute the statistical tools that are employed to study a data set.

The tools allow an analyst to to build a framework for good decision making.
\newline
def 2: \textbf{Statistics} are mathematical quantities or expressions that are applied on data in order to elicit essential characteristics of the data at hand e.g the statistic for obtaining central value of the sample on the mean given by $\bar{x} = \frac{1}{n}\sum_{i = 1}^{n} x_i$ where $i = 1,2,3,4, \dots,19$ are sample parts.

def 3: \textbf{Statistics} may also imply the value calculated in def 2.
The third definition is ussually less found or used in formal context.
\newpage
%end of chapter
