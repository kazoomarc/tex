\chapter{Measures of Central Tendancy}%!chapter

\section{Median}
The median is a measure of location or centrality of the observations, at $50$th percentile or second qaurtile.

other commonly used measures of central tendency are mode and mean.
\section{Mode}
The mode of the data set is the value that occurs most.
\section{Mean}

\textbf{Definition: }If $X$ is a sample of data from some population and if the sample size is $n$ then,


\begin{equation}
    \bar{x} = \frac{\sum_{i = 1}^{n} x_i}{n}= \frac{x_1 + x_2 + x_3 + \ldots + x_n}{n}
\end{equation}

Is called sample mean, when $x$ are called sample observation ($x \implies$ variable, $x \impliedby$ value)

\textbf{Definition: }If $x$ is a variable for the population of data where at total population has $N$ observations and $x$, are measurements for each individual then;
\begin{equation}
    \mu = \frac{\sum_{i = 1}^{N} x_i}{N}= \frac{x_1 + x_2 + x_3 + \ldots + x_N}{N}
\end{equation}

eg: For the dataset we used in previous example.

\begin{align*}
    Median  & = 16                                                                                                \\
    mode    & = 16                                                                                                \\
    mean    & = \bar{x}                                                                                           \\
    \bar{x} & = \frac{\sum_{i = 0}^{20}x_i }{20}                                                                  \\
    \bar{x} & = \frac{(6 + 9 + 10 + 12 +13 + 14(2) +15 + 16(3) + 17 + 17 + 18 + 18 + 19 + 20 + 21 + 22 + 24)}{20}
\end{align*}


Interpretation: The distribution is concentrated around value $15.85$

\newpage
%end of chapter
