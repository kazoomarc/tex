\chapter{Measures of Position}%!chapter

\section{Percentiles}
The $P_{th}$ percentile of a group of numbers is that value below which lies $P\%$ (P percent) of the numbers in the group.

The Position of the $P_{th}$ percentile is given by
\begin{equation}
    P_{th} = \frac{(n+1)p}{100}
\end{equation}

Where $n$ is the number of data points.

Example:

A department store's data on sales made by each  of its sales person. The data, number of sales made on a given data by each of $20$ sales persons are as follows
\begin{table}[H]
    \centering
    \caption{Data on sales}
    \begin{tabular}{|*{20}{c|}}

        \hline
        9  & 6  & 12 & 10 & 13 & 15 & 16 & 14 & 14 & 16 &
        17 & 16 & 24 & 21 & 22 & 18 & 19 & 18 & 20 & 17   \\
        \hline
    \end{tabular}
\end{table}


Find the $50_{th}$, $80_{th}$ and $90_{th}$ percentiles of this data set.

Solution:
\begin{table}[H]
    \centering
    \caption{Data on sales Sorted}
    \begin{tabular}{|*{20}{c|}}

        \hline
        6 & 9 & 10 & 12 & 13 & 14 & 14 & 15 & 16 & 16 & 16 & 17 & 17 & 18 & 18 & 19 & 20 & 21 & 22 & 24 \\
        \hline
    \end{tabular}
\end{table}

$n = 20$

\begin{enumerate}
    \item[(a)]50th Percentile \\ $P_{th}$ percentile $= \frac{(n + 1)p}{100}$\\
    $therefore$ $50_{th}$ $= \frac{(20 + 1)50}{100}$
    $= 10.5$

    The value at position $10.5$ is the same as the value at the middle of positions $10$ and $11$, which is the average of $16$ and $16$

    Hence $50_{th}$ percentile is $16$

    Interpretation: This implies that $50\%$ of the given data points are below the value $16$.
    \item[(b)] 80th Percentile = $\frac{(20 + 1)\times 80}{100}$ \\ $= 16.8$

        The $16$th observation is $19$ and observation at position $17$ is $20$. Therefore $80$th percentile is a point lying $0.8$ of the way from $19$ to $20$, that is $19.8$

        Interpretetion: The $80$th value of 19.8 means that $80\%$ of the dat are below $19.8$
\end{enumerate}

\section{Quartiles}
Quartiles are the percentage points that break down the dataset into quarters.

First quarter, Second quarter, third quarter and fourth quarter.

The first Quartile is the 25th percentile. It is the point below which lie one-fourth of the data.

Similary, the second quarterile is the 50th percentile. This is the most important point and has a special name , it is the \textbf{median}

Definition: The median is a point below which lie half the data. it is the 50th percentile.

The third quartile is the 75th percentile point, it is that point below which lie $75\%$ of the data.

NB: 25th percentile is called lower quartile, 50th percentile is called the middle quartile. 75th percentile is called the upper quartile.

Example:

Using data from previous example find lower, middle and Upper Quartiles

$Q1$ \\
$\frac{(20 + 1)25}{100} = 5.25$

Q1 is the value at position $5.25$ \\
The value at position $5$ is $13$ and the one at $6$ is $14$

Therefore;

\begin{align*}
    Q1 & = 13 + 0.25 *(14 - 13) \\
    Q1 & = 13 + 0.25            \\
    Q1 & = 13.25                \\
\end{align*}

$Q2$

$\frac{(20 + 1)50}{100} = 10.5$

Q1 is the value at position $10.5$ \\
The value at position $10$ is $16$ and the one at $6$ is $16$

Therefore;

\begin{align*}
    Q2 & = 16 + 0.5 *(16 - 16) \\
    Q2 & = 16 + 0              \\
    Q2 & = 16                  \\
\end{align*}
$Q3$
$\frac{(20 + 1)75}{100} = 15.75$

Q1 is the value at position $15.75$ \\
The value at position $15$ is $18$ and the one at $16$ is $19$

Therefore;

\begin{align*}
    Q3 & = 18 + 0.75 *(19 - 18) \\
    Q3 & = 18 + 0.75            \\
    Q3 & = 18.75                \\
\end{align*}

\newpage
%end of chapter
