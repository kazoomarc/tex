\chapter{Set Theory}%!chapter

\section{Sets and Elements}
\begin{definition}
    A set is a well defined list or collection of objects.
\end{definition}

\begin{definition}
    An object that belongs to a particular set is called an element or a member of that set.
\end{definition}

\section{Set Notations:}

\begin{para}
    Uppercase letters of the alphabet, i.e., $A, B, M, X$ are used to denote sets and lowercase letters i.e., $a, b, m, x$ for members of sets
\end{para}

\begin{para}
    if $A$ is a set and $k$ its element, then the expression $k \in A$ is read as "$k$ is an element of $A$". If $p$ is not a member of $A$ it is written as $p \notin A$
\end{para}

\subsection{Examples of Sets}

\begin{para}
    To specify that certain objects belong to a given set, the braces, \{\} called set builders are used by either providing a \underline{roster} i.e., complete list of all elements in the braces or the \underline{rule} method i.e., stating properties that characterize the elements within the braces
\end{para}

\begin{examples}
    \mbox{}\\[-\baselineskip] %empty line
    \begin{enumerate}
        \item $A = \{1,2,3,5,7,11\}$ means is a set consisting of numbers $1,2,3,5,7,$ and $11$
        \item $B = \{x:$ x is a prime number; $x < 13\}$ means $B$ is a set of prime numbers less than $13$
        \item $17 \notin A$ means $17$ is not a member of set $A$
    \end{enumerate}
\end{examples}

\section{Subset}

\begin{definition}
    Let $A$ and $B$ be two sets. If every element of $A$ also belongs to $B$ i.e, if $p \in A \implies p \in B$,then $A$ is called a subset of $B$ or is said to be contained in $B$.
\end{definition}

\begin{para}
    Two different types of subsets emerge from this definition of subset: \underline{proper} and \underline{improper} subsets.
\end{para}

\begin{definition}
    A is a proper subset of $B$, if it is a subset of $B$ and there exists at least one element of $B$ that does not belong to $A$. Otherwise $A$ is an improper subset of $B$.
\end{definition}

\section{Subset Notation}

\begin{para}
    To show that $A$ is a subset of $B$ we write it as $A \subset B$.
\end{para}

\subsection{Subset Examples}

let:\\
$\mathbb{N} = \{1, 2, 3, \ldots\}$\\
$\mathbb{Z} = \{\ldots, -2, -1, 0, 1, 2, \ldots\}$\\
$\mathbb{R} = \{x: \text{ x is a real number}\}$\\

Then $\mathbb{N} \subset \mathbb{Z} \subset \mathbb{R}$

\begin{para}
    A subset of real numbers can also be defined as an interval on the real line.

    \begin{center}
        \includegraphics[width=0.3\linewidth]{chapter/sets and elements/numberLine.png}
    \end{center}

\end{para}


\begin{examples}
    \mbox{}\\[-\baselineskip] %empty line
    \begin{enumerate}
        \item $(a,b) = \{x:  a<x<b, x \in \mathbb{R}\} $\\
              means an open interval of real numbers from $a$ to $b$.
        \item $[a,b] = \{x: a \leq x \leq b, x \in \mathbb{R}\} $\\
              means $a$ closed interval of real numbers from $a$ to $b$.
        \item $(a,b] = \{x: a < x \leq b, x \in \mathbb{R}\} $\\
              means an open-closed interval of real numbers from $a$ to $b$.
        \item $[a,b) = \{x: a \leq x < b, x \in \mathbb{R}\} $\\
              means an closed-open interval of real numbers from $a$ to $b$.
    \end{enumerate}
\end{examples}
\section{Equal Sets}
\begin{definition}
    Two sets are equal if each set is contained in another i.e., let $A$ and $B$ be any two sets, then if $x \in A \implies x \in B$ and $x \in B \implies x \in A$, we have $A = B$
\end{definition}

\begin{para}
    In other words $A=B$ if and only if $A \subset B$ and $B \subset A$
\end{para}
\begin{para}
    The Definition $A \subset B$ includes the case $A=B$. If $A \subset B$ but $A \ne B$ then we say that $A$ is a \underline{proper subset} of $B$
\end{para}

\begin{para}
    Eg., let $A = \{1,w,10\}$ and $B = \{w,10,1\}$. Since all elements of A are also in $B$ and vice versa, then $A = B$
\end{para}
\section{Universal Set and Empty Set}

\begin{para}
    When in a certain discussion we have all the sets under analysis i.e., $A,B,C$ etc. being subset of one set that we denote by $U$, then $U$ is called \underline{the Universal set}
\end{para}

\begin{para}
    A null or empty set denoted by $\emptyset$ is a set with no elements i.e, $\emptyset = \{\}$
\end{para}

\begin{para}
    An empty set is a subset of every other set $A$ i.e $\emptyset \in A$
\end{para}

\begin{para}
    This is the case because "every $x \in \emptyset$ (there are none) also belongs to $A$" is a true statement for any set $A$ since there is no $x \in \emptyset$ to make the statement false.
\end{para}

\begin{para}
    Eg, Let $C = \{x:x^2 = 4$ is odd member$\}$. Then $C = \emptyset$
\end{para}

\begin{theorem}
    :\\Let $A, B$ and $C$ be any sets then
    \begin{enumerate}
        \item $A \subset A$
        \item if $A \subset B$ and $B \subset A$ then $A = B$
        \item if $A \subset B$ and $B \subset C$ then $A \subset B$
    \end{enumerate}
\end{theorem}


\begin{proofs}
    :\\
    \begin{enumerate}
        \item let $x \in 1^{st} A$, then by uniqueness of sets $x \in 2^{nd} A$. Hence by the definition of subset $A \subset A$
        \item let $x \in A$. Then $A \subset B \implies x \in B$(by definition of subset) likewise, $B \subset A$ if $y \in B$ then $y \in A$. now we have $x \in A \implies x \in B$ AND $y \in B \implies y \in A$ hence $x = y$ i.e., $A = B$
        \item let $x \in A$, then $A \subset B \implies x \in B$. while $B \subset C \implies x \in C$ $\therefore A \subset B$ and $x \in C$, Hence $A \subset B \subset C$ or $A \subset C$
    \end{enumerate}
\end{proofs}
\section{Set Operation}
\subsection{Union}
\begin{para}
    Let $A$ and $B$ be any sets. The union of $A$ and $B$ denoted by $A \cup B$ is the set that consists of all elements that belong to $A$ or to $B$ or to both, i.e., $A \cup B = \{x:x \in A \text{ or } x \in B\}$
\end{para}
\begin{para}
    let $A = \{1,2,3,4\}$ and $B = \{3,4,5,6\}$. Then $A \cup B = \{1,2,3,4,5,6\}$
\end{para}
\subsection{Intersection}
\begin{para}
    Let $A$ and $B$ be any sets. The intersection of $A$ AND $B$ denoted by $A \cap B$, is the set of elements which belong to both A AND B i.e, $A \cap B = \{x:x \in A, x \in B\}$
\end{para}

\begin{para}
    If $A \cap B = \emptyset$, that is if $A$ and $B$ do not have elements in common, then $A$ and $B$ are said to be disjoint.
\end{para}

\begin{para}
    Eg: From the previous example, $A = \{1,2,3,4\}$, $B = \{3,4,5,6\}$ Then $A \cap B = \{3,4\}$
\end{para}

\subsection{Complement}
\begin{para}
    Let $A$ be any set and $U$, the Universal set, the complement of $A$, denoted by $A^c$, is the set of elements which do not belong to $A$, i.e $A^c = \{x:x \in U, x \in A\}$

    E.g: \\
    Let $A = \{1,2,3,4\}$ and $U = \{1,2,3, \dots\}$. Then $A^c = \{5,6,7,\dots\}$
\end{para}

\subsection{Difference}
\begin{para}
    let $A$ and $B$ be any sets. the difference of $A$ and $B$ or \underline{relative difference} of $B$ with respect to $A$, denoted by $A\backslash B$, is the set of elements which belong to $A$ but not to $B$.\\
    i.e., $A\backslash B$ = $\{x:x \in A, x \notin B\}$
\end{para}


\begin{center}
    \includegraphics[width=0.5\linewidth]{chapter/sets and elements/intersection.png}
\end{center}


\begin{para}
    Note that $A\backslash B$ AND $B$ are disjoint i.e, $(A\backslash B) \cap B = \emptyset$
\end{para}

\begin{para}
    E.g: Using the previous example, with
    \begin{align*}
        A             & = \{1,2,3,4\} \\
        B             & = \{3,4,5,6\} \\
        A\backslash B & = \{1,2\}     \\
    \end{align*}

\end{para}


\subsection{Symmetric Difference}

\begin{para}
    The Symmetric difference of the sets $A$ and $B$, denoted by $A \oplus B$ is the set consisting of those elements which belong to $A$ or $B$, but not both i.e, $A \oplus B = (A \cup B)\backslash (A \cap B)$ or $A \oplus B = (A \backslash B)\nobreakspace U \nobreakspace(B\backslash A)$
    E.g: If
    \begin{align*}
        A          & =\{1,2,3,4\}                       \\
        B          & =\{3,4,5,6\}                       \\
        A \oplus B & = (A \cup B)\backslash (A \cap B)  \\
                   & =\{1,2,3,4,5,6\}\backslash \{3,4\} \\
                   & = \{1,2,5,6\}                      \\
    \end{align*}
\end{para}

\subsection{Product Set or Cartesian Product}
\begin{para}
    An $n$-tuple is an ordered array on $n$ written $(x_1,x_2,\ldots,x_n)$
\end{para}
\begin{para}
    $(1,2), (0, 100), (a, b)$ are examples of 2-tuples, while $(1,1,1), (a,c,b), (2,1,2)$ are cases of 3-tuples.
\end{para}

\begin{para}
    let $A$ and $B$ be two sets. The product set or cartesian product of $A$ and $B$, denoted by $A \times B$ is a set of all possible 2-tuples $(a,b)$, where $a \in A$ and $b \in B$, i.e. $A \times B = \{(x,y): x \in A, y \in B\}$ e.g. \\
    let:\\
    $A=\{1,2,3\}$ and\\
    $B=\{a,b\}$, Then \\
    $A \times B = \{(1,a),(1,b),(2,a),(2,b),(3,a),(3,b)\}$ while\\
    $B \times A = \{(a,1),(a,2),(a,3),(b,1),(b,2),(b,3)\}$
\end{para}

\section{Properties of Set Operations}


\subsection*{1. Idempotent Laws}
\begin{enumerate}
    \item $A \cup A = A$
    \item $A \cap A = A$
\end{enumerate}

\subsection*{2. Associative Laws}
\begin{enumerate}
    \item $(A \cup B) \cup C = A \cup (B \cup C) $
    \item $(A \cap B) \cap C = A \cap (B \cap C) $
\end{enumerate}


\subsection*{3. Commutative Laws}
\begin{enumerate}
    \item $ A \cup B = B \cup A$
    \item $ A \cap B = B \cap A$
\end{enumerate}


\subsection*{4. Distributive Laws}
\begin{enumerate}
    \item $A \cup (B \cap C) = (A \cup B)\cap(A \cup C)$
    \item $A \cap (B \cup C) = (A \cap B)\cup(A \cap C)$
\end{enumerate}

\subsection*{5. Identity Laws}
\begin{enumerate}
    \item $A \cup \emptyset = A$
    \item $A \cup U = U$
    \item $A \cap U = A$
    \item $A \cap \emptyset = \emptyset$
\end{enumerate}

\subsection*{6. Complement Laws}
\begin{enumerate}
    \item $A \cup A^c = U$
    \item $A \cap A^c = \emptyset$
    \item $(A^c)^c = A$
    \item $U^c = \emptyset$
    \item $\emptyset^c = U$
\end{enumerate}
\subsection*{7. De Morgan's Laws}
\begin{enumerate}
    \item $(A \cup B)^c = A^c \cap B^c$
    \item $(A \cap B)^c = A^c \cup B^c$
\end{enumerate}
\section{Finite sets, Classes of sets}
\subsection{Finite set}

A set $A$ is finite if $A$ is empty or $A$ consists of exactly $m$ elements, where $m$ is a positive integer otherwise $A$ is an \underline{infinite}.
e.g.

Let $M$ be a set of days of a week i.e. $M = \{Mon, Tues, Wed,Thurs,Fri,Sat, Sun\}$ $\therefore M$ is a finite set.

Let $W = \{x: x$ is a river on earth$\}$, Although it may be difficult to count the number of rivers on earth, with time this can be done hence $W$ is a finite set.

Let $I$ be a unit interval of real numbers, i.e. $I = \{x: 0 \leq x \leq 1\}$ then $I$ is an infinite set.

Let $Y$ be a set of positive even integers, i.e $Y = \{2,4,6,\ldots\}$ Then $Y$ is an infinite set.
\subsection{Countable set}

A set $A$ is countable if $A$ is finite or if $A$ has elements that can be arranged in the form of a sequence in this case $A$ is said to be \underline{countably infinite} otherwise $A$ is \underline{uncountable}.
\vspace*{12pt}

\textbf{Counting elements in finite sets}
Let $A$ be any set, then the notation $n(A)$ shall denote the number of elements in $A$. e.g if $A = \{1,a,w,8,4\}$ then $n(A) = 5$

For an empty set at $\emptyset$, $n(\emptyset) = 0$

\begin{lemma}
    :\\
    a). Suppose $A$ and $B$ are disjoint sets. Then $A \cup B$ is finite and hence $n(A \cup B) = n(A) + n(B)$

    b). $n(A \backslash B) = n(A) - n(A \cap B)$

    c). If $U$ is a universal set, then $n(A^c) = n(U) - n(A)$
    d). Suppose $A$ and $B$ are finite then $A \times B$ is finite and $n(A \times B) = n(A) \times n(B)$
\end{lemma}

\begin{theorem}[Inclusion exclusion-principle]
    Suppose $A$ and $B$ are finite sets. Then $A \cap B$ and $A \cup B$ are finite and $n(A \cup B) = n(A) + n(B) - n(A \cap B)$

    This can be generalized to any finite number of finite sets e.g. if $A$, $B$ and $C$ are finite sets then $A \cup B \cup C$ is finite and $n(A \cup B \cup C) = n(A) + n(B) + n(C) - n(A \cap B) - n(A \cap C) + n(A \cap B \cap C)$
\end{theorem}

Example:

Suppose in form $3$ class, $30$ students take maths and $35$ take English, while $20$ take both subjects. Find the number that study.

a. only maths \\
b. Maths and English\\
c. exactly one of the two subjects

Solution:

Let $M$ be a list of students taking Maths and for $E$ for English.

a).\begin{align*}
    n(M \backslash E) & = n(M) - n(E) \\
                      & = 30 - 20     \\
                      & =10           \\
\end{align*}

b). \begin{align*}
    n(M or E) & = n(M \cup E)               \\
              & = n(M) + n(E) - n(M \cap E) \\
              & = 30 + 35 - 20              \\
              & = 25                        \\
\end{align*}

c). \begin{align*}
    n(M \oplus E) & = n[(M \backslash E) \cup (E \backslash M)]   \\
                  & = n(M \backslash E) + n(E \backslash M)       \\
                  & = [n(M) - n(M \cap N)] + [n(E) - n(M \cap E)] \\
                  & = 30 - 20 + 35 - 20                           \\
                  & = 25                                          \\
\end{align*}

\subsection{Classes of Sets}

Member of a set can be sets themselves. To help in clarifying these situations we usually use the term \underline{class} or \underline{family} for such a set. The words subclass and \underline{subfamily} have meanings.

Analogous to subset

e.g.

let $S = \{1,2,3,4\}$ let $W$ be the class of subsets of $S$ which contain exactly three elements of $S$. Then $W = [\{1,2,3\}, \{1,2,4\}, \{1,3,4\}, \{2,3,4\}]$


\subsection{Power set}

Consider any set $A$ the power set of $A$, denoted by $P(A)$ is the class of all subsets of $A$.

In general of $A$ is finite, so is $P(A)$.

The number of elements in $P(A)$ is $2^{n(A)}$ i.e, $n(P(A)) = 2^{n(A)}$


e.g.

Let $A = \{a,b,c\}$. Then $P(A) = \{a,b,c\}, \{a,b\} , \{b, c\}, \{a,c\}, \{a,c\}, \{a\}, \{b\}, \{c\}, \{\emptyset\}$

\subsection{Partitions}

Let $X$ be a non-empty set. A partition of $X$ is a subdivision of $X$ into non overlapping, non empty subsets. The subsets in a partition are called cells.

\begin{center}
    [[[[[[[[[[[[[[[[Image image image]]]]]]]]]]]]]]]]
\end{center}

E.g. If $X = \{1,2,3, \ldots\}$ Then $W = [\{1,2,3,4\}, \{8,7\}]$ is a partition of $X \{5,6,7\}$